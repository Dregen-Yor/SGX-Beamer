\documentclass[hyperref={bookmarks=false},aspectratio=169, usenames,dvipsnames, hideothersubsections]{beamer}
\usepackage{Dependency}
%\setbeamertemplate{footline}[frame number]

% ------------  Information on the title page  --------------------
\title[SGX-Beamer] % This appears on the sidebar and footer 这里在侧边栏和页脚显示
{\bfseries{SGX-Beamer}} % Title of the presentation, appears on the title slide 在主页显示

\subtitle{测试副标题} % Subtitle appears in the slide

\author[孙更欣] % This appears on the side bar
{孙更欣\inst{1}}

\institute[SDU]  % This appears on footer
{
  \inst{1}
  计算机科学与技术学院\\ % This appears in the slide
  山东大学
}

\date[\today] % Purpose and Date of the Presentation, appears on the footer
{Purpose of the Presentation and/or Venue} %% Appears in the slide
%------------------------------------------------------------

%------------------------------------------------------------
%The next block of commands puts the table of contents at the 
%beginning of each section and highlights the current section:

\AtBeginSection[]
{
  \ifthenelse{\boolean{sectiontoc}}{
    \begin{frame}%[noframenumbering]
      \frametitle{章节预览}
      \Large\tableofcontents[sections=\value{section}]
      \normalsize
    \end{frame}
  }
}
%------------------------------------------------------------

\newcommand{\toclesssection}[1]{
  \setboolean{sectiontoc}{false}
  \section{#1}
  \setboolean{sectiontoc}{true}
}

\begin{document}

\frame{\titlepage}  % Creates title page

%---------   table of contents after title page  ------------
\begin{frame}
\frametitle{Presentation Overview}
\tableofcontents
\end{frame}
%---------------------------------------------------------


\section{Subtopic 1}
%---------------------------------------------------------
%Changing the visibility of the text
\begin{frame}
\frametitle{Details under subtopic 1}
测试中文119$119$ 。 % Testing Chinese characters
\begin{itemize} % The slides will keep adding one point per slide
    \item<1-> Some details to provide context. 
    \item<2-> Further details followed by the previous context.
    \item<3-> Some examples followed by the previous information.
\end{itemize}
\end{frame}

\subsection{Subtopic 1.1}
\begin{frame}
\frametitle{Details under subtopic 1.1}
测试一下子目录。
\end{frame}
%---------------------------------------------------------


%---------------------------------------------------------
%\begin{frame}  % Example of the \pause command
%This slide is to test mathematical formulas \pause
%
%$$E=mc^2$$ \pause
%
%as well as the ``pause'' functionality
%\end{frame}
%---------------------------------------------------------

\section{Subtopic 2}

%---------------------------------------------------------
%Highlighting text
\begin{frame}
\frametitle{Subtopic 2}

This is a brief introduction of \alert{Subtopic 2}. % Alert makes things prominent

\begin{block}{Terminology}
Some definitions for the terminology are here.
\end{block}

\begin{alertblock}{Context}
Some information to provide context here.
\end{alertblock}

\begin{examples}
See the next slide for a two-column example.
\end{examples}
\end{frame}
%---------------------------------------------------------


%---------------------------------------------------------
%Two columns
\begin{frame}
\frametitle{Two Column Section}

\frametitle{Two Column Section}

\begin{columns}

\column{0.45\textwidth}
Lorem Ipsum is simply dummy text of the printing and typesetting industry. Contrary to popular belief, Lorem Ipsum is not simply random text. It has roots in a piece of classical Latin literature from 45 BC, making it over 2000 years old. 

\column{0.55\textwidth}
Richard McClintock, a Latin professor at Hampden-Sydney College in Virginia, looked up one of the more obscure Latin words, consectetur, from a Lorem Ipsum passage, and going through the cites of the word in classical literature, discovered the undoubtable source.
\end{columns}

It has survived not only five centuries, but also the leap into electronic typesetting, remaining essentially unchanged.

\end{frame}
%---------------------------------------------------------


\end{document}